\section{Conclusions}

The FeedApp project has been a valuable learning experience in understanding and comparing different authentication technologies, specifically OAuth 2.0 and Basic Authentication. The goal was to evaluate these methods in terms of scalability, security, and how easy or difficult they are to implement. By exploring these technologies through practical experiments, the project has shown how important it is to choose the right authentication method for an application, depending on its requirements.

OAuth 2.0 proved to be a powerful framework, offering advanced features like token expiration, scope-based access control, and stateless validation. These features make it highly secure and scalable, especially for applications that handle a lot of users or sensitive data. However, OAuth 2.0 is more complex to set up and has a steeper learning curve. It took more time and effort to implement, but the detailed documentation and community support made the process manageable.

On the other hand, Basic Authentication was much easier and quicker to implement. Its simplicity makes it a good option for smaller applications or systems that don’t have high security or scalability demands. However, during testing, it struggled to handle large numbers of users and lacked the advanced features that modern applications often need. This shows that Basic Authentication works well for simpler use cases but isn’t suitable for high-demand systems.

One of the key takeaways from this project is how important it is to choose the right technology for the job. OAuth 2.0 is clearly a better choice for systems that need to handle growth and protect sensitive data, while Basic Authentication might work fine for smaller, less demanding applications. It also became clear that the maturity and widespread use of a technology, like OAuth 2.0, make it a reliable and future-proof choice for modern applications.

The experiments and comparisons carried out during this project have provided a solid understanding of these technologies, their strengths, and their weaknesses. While Basic Authentication is simple and fast, OAuth 2.0 stands out as a more capable and secure solution for the kind of features and user loads that this FeedApp was designed to handle. This experience will prove helpful in future projects where authentication plays an important role.