\section{Conclusions}

The FeedApp project has been a valuable learning experience in understanding and comparing different authentication technologies, specifically OAuth 2.0 and Basic Authentication. The goal was to evaluate these methods in terms of scalability, security, and how easy or difficult they are to implement. By exploring these technologies through practical experiments, the project has shown how important it is to choose the right authentication method for an application, depending on its requirements.

OAuth 2.0 stood out as a powerful and modern authentication framework. It provides advanced security features like token expiration, scope limitations, and stateless validation, which make it very scalable and suitable for large systems. However, implementing OAuth 2.0 requires more effort and a steeper learning curve, especially for someone new to the framework. Thankfully, its documentation is thorough and supported by an active community, which helped during the project. In comparison, Basic Authentication is much simpler and quicker to set up, making it a good choice for small applications or systems that don’t handle a lot of users. However, it doesn’t scale well under heavy loads and lacks the advanced security features that OAuth 2.0 offers.

The process of working on this project also highlighted how the maturity of a technology impacts its usability. OAuth 2.0 has evolved over the years and is now widely used in the industry, making it a reliable and future-proof option for modern applications. In comparison Basic Authentication feels outdated for applications that require strong security and scalability.

The references used throughout the project, including the official OAuth 2.0 specification and documentation from Spring Security, were essential in understanding the technologies and their practical applications. These resources, along with case studies and comparisons from the industry, offer a strong foundation for anyone interested in learning more about authentication.

Overall, this project has shown that while Basic Authentication is simple and straightforward, OAuth 2.0 is the better choice for applications that need to scale or handle sensitive data securely. The practical experience gained through implementing and testing these technologies has provided valuable insights into their strengths and limitations, which will be useful for similar projects in the future.