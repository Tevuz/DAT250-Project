\section{Introduction}
\label{sec:introduction}

% A brief introduction to the prototype implementation and topic of the project.
The FeedApp project seeks to develop a running prototype of a full-stack FeedApp. The purpose of this application is to allow users to create and manage polls and surveys. Registered users can also vote on polls using a web-based user interface. 
\medskip

% Discuss (briefly) the technology stack that has been selected
\noindent
The technology stack is as follows:
\begin{itemize}
\item \textbf{Java with SpringBoot}: Serving as the core platform and framework, Spring Boot provides the foundation for building a lightweight and production-ready application.
\item \textbf{Spring Web MVC}: This framework enables the development of a RESTful HTTP API, facilitating communication between the front-end and back-end services.
\item \textbf{Svelte + Vite}: An innovative and modern approach to front-end development, where Svelte offers a highly reactive and efficient framework for building Single Page Applications (SPAs), and Vite ensures fast build times and optimized development workflows.
\item \textbf{Jakarta Persistence API with Hibernate and Postgres}:  For object-relational database mapping, JPA with Hibernate will interact with a PostgreSQL database to manage structured data such as users, surveys, polls, and votes.
\item \textbf{MongoDB}: A non-relational database to store and manage unstructured or semi-structured data, complementing the relational data model.
\item \textbf{RabbitMQ}: A message broker facilitating asynchronous communication between various components of the application, ensuring scalability and reliability.
\item \textbf{Docker}: A containerization engine to streamline development, deployment, and orchestration of the application across different environments.
\item \textbf{OAuth 2.0}: Ensures secure user authentication and authorization, providing a scalable and standard mechanism to handle access control. Further in this report we conduct a technology assessment according to Brown and Walnau.\cite{brown:96} based on this technology.
\end{itemize}

% A brief account of the results that have been obtained in the project.
\noindent
 *A brief account of the results that have been obtained in the project.*
\medskip

% A one paragraph overview at the end, explaining how the rest of the report is / has been organised.
\noindent
This rest of this report is organised as follows:

\smallskip

\noindent
Section~\ref{sec:design} gives a detailed descprition about the functional aspects of the FeedApp application. This includes use cases, domain model and an application flow diagram.
\smallskip

\noindent
Section~\ref{sec:technology} documents the experiments conducted, following the methodology described in the paper by Brown and Walnau.
\smallskip

\noindent
Section~\ref{sec:implementation} gives a brief explanation of how the prototype has been implemented. It provides details to those who may want to run the application and how they may develop it further.