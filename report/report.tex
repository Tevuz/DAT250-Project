\documentclass[11pt]{article}

\usepackage{a4wide}
\usepackage{mathptm}
\usepackage{xspace}
\usepackage{amsmath}
\usepackage{graphicx}
\usepackage{algorithm}
\usepackage{algpseudocode}
\usepackage{tikz}
\usepackage{tkz-graph}
\usetikzlibrary{shapes.misc, positioning}
\usepackage{listings}
\usepackage{color}
\usepackage{hyperref}

\definecolor{dkgreen}{rgb}{0,0.6,0}
\definecolor{gray}{rgb}{0.5,0.5,0.5}
\definecolor{mauve}{rgb}{0.58,0,0.82}

\lstset{frame=tb,
  language=Java,
  aboveskip=3mm,
  belowskip=3mm,
  showstringspaces=false,
  columns=flexible,
  basicstyle={\small\ttfamily},
  numbers=left,
  numberstyle=\tiny\color{gray},
  keywordstyle=\color{blue},
  commentstyle=\color{dkgreen},
  stringstyle=\color{mauve},
  breaklines=true,
  breakatwhitespace=true,
  tabsize=3
}
\begin{document}

\title{My Best Software Technology Evaluation Project Ever}

\author{Kristian E. Takvam and Lars Magnus Nordeide}

\maketitle

\begin{abstract}
%Abstract: 10-15 lines with the software technology and the highlights from the project that has been undertaken.
This project focuses on the design and implementation of FeedApp, an interactive web application for creating and participating in surveys and polls. Built with a modern technology stack, the project highlights the integration of OAuth 2.0 for secure user authentication and token-based authorization, ensuring a seamless and secure experience for end-users. The backend leverages Spring Boot and Spring Web MVC to provide a robust RESTful API, while the frontend, developed using Svelte + Vite, delivers a fast and responsive Single Page Application (SPA).

The application is underpinned by a well-defined domain model, capturing the relationships between key entities such as users, surveys, polls, and votes. Data persistence is managed using PostgreSQL and MongoDB, accommodating structured and unstructured data needs, while RabbitMQ ensures efficient message-driven communication between services. The system is containerized using Docker, enabling smooth deployment and scalability.

A technology assessment was conducted to validate the decision to implement OAuth 2.0, comparing it to alternatives like API Keys and Basic Authentication. The assessment highlights OAuth 2.0's superior scalability, security, and flexibility, making it a suitable choice for modern web applications.This project demonstrates how a modular, scalable architecture can enable collaborative participation in surveys and polls in real-time, while ensuring robust security and an intuitive user experience.

\end{abstract}

%\input{commands}

\section{Introduction}
\label{sec:introduction}

% A brief introduction to the prototype implementation and topic of the project.
The FeedApp project seeks to develop a running prototype of a full-stack FeedApp. The purpose of this application is to allow users to create and manage polls and surveys. Registered users can also vote on polls using a web-based user interface. 
\medskip

% Discuss (briefly) the technology stack that has been selected
\noindent
The technology stack is as follows:
\begin{itemize}
\item \textbf{Java with SpringBoot}: Serving as the core platform and framework, Spring Boot provides the foundation for building a lightweight and production-ready application.
\item \textbf{Spring Web MVC}: This framework enables the development of a RESTful HTTP API, facilitating communication between the front-end and back-end services.
\item \textbf{Svelte + Vite}: An innovative and modern approach to front-end development, where Svelte offers a highly reactive and efficient framework for building Single Page Applications (SPAs), and Vite ensures fast build times and optimized development workflows.
\item \textbf{Jakarta Persistence API with Hibernate and Postgres}:  For object-relational database mapping, JPA with Hibernate will interact with a PostgreSQL database to manage structured data such as users, surveys, polls, and votes.
\item \textbf{MongoDB}: A non-relational database to store and manage unstructured or semi-structured data, complementing the relational data model.
\item \textbf{RabbitMQ}: A message broker facilitating asynchronous communication between various components of the application, ensuring scalability and reliability.
\item \textbf{Docker}: A containerization engine to streamline development, deployment, and orchestration of the application across different environments.
\item \textbf{OAuth 2.0}: Ensures secure user authentication and authorization, providing a scalable and standard mechanism to handle access control. Further in this report we conduct a technology assessment according to Brown and Walnau.\cite{brown:96} based on this technology.
\end{itemize}

% A brief account of the results that have been obtained in the project.
\noindent
 *A brief account of the results that have been obtained in the project.*
\medskip

% A one paragraph overview at the end, explaining how the rest of the report is / has been organised.
\noindent
This rest of this report is organised as follows:

\smallskip

\noindent
Section~\ref{sec:design} gives a detailed descprition about the functional aspects of the FeedApp application. This includes use cases, domain model and an application flow diagram.
\smallskip

\noindent
Section~\ref{sec:technology} documents the experiments conducted, following the methodology described in the paper by Brown and Walnau.
\smallskip

\noindent
Section~\ref{sec:implementation} gives a brief explanation of how the prototype has been implemented. It provides details to those who may want to run the application and how they may develop it further.

\section{Design}
\label{sec:design}

% Around 5 pages about functional aspects of the FeedApp application.
The main purpose of the FeedApp is to deliver a modern web application that offers a seamless and intuitive experience, allowing users to log in, craft their own surveys and cast votes on existing polls.

\subsubsection*{Use cases}
Blah blah use cases..

\subsubsection*{Domain model}
\begin{figure}[thb]
	\centering
	\includegraphics[scale=0.5]{figs/domainmodel.png}
	\caption{The Domain Model.}
	\label{fig:domainmodel}
\end{figure}

\subsubsection*{Architecture}
Blah blah architecture..


%You may have a look at the \href{https://github.com/selabhvl/dat250public/blob/master/projectdescription/README.md}{Examples on GitHub}.



\section{Technology Assessment}
\label{sec:technology}

Introduce in (sufficient) depth the key concepts and architecture of the chosen software technology. As part of this, you may consider using a running example to introduce the technology.

This part and other parts of the report probably need to refer to
figures. Figure~\ref{fig:framework} from \cite{brown:96} just
illustrates how a figure can be included in the report.

\begin{figure}[thb]
	\centering
	\includegraphics[scale=0.5]{figs/framework.png}
	\caption{Software technology evaluation framework.}
	\label{fig:framework}
\end{figure}

\subsection{Descriptive Modeling}
%write where the technology comes from, its history, its context and what problem it solves.
%Consider drawing a graph like in \cite{brown:96}.
OAuth 2.0 is a modern framework designed to facilitate secure access to resources on behalf of a user. It achieves this through a token-based system that delegates access using short-lived Access Tokens and optional Refresh Tokens. These tokens are associated with specific scopes, allowing fine-grained authorization to resources without exposing user credentials. OAuth 2.0's adoption as a web standard ensures compatibility with a wide range of APIs and systems, making it an ideal choice for securing FeedApp’s RESTful services.

API Keys represent an older, simpler approach to authenticating clients. An API Key is a static string generated by the server and assigned to a client. The client includes the key in each request to authenticate itself. While API Keys are easy to implement, they lack features such as token expiration, scope management, and secure delegation, making them less suitable for modern web applications.

Basic Authentication is a legacy method of authentication that sends a client’s username and password in every request’s header. While easy to implement, Basic Authentication poses significant security risks, as credentials are repeatedly transmitted and stored. It also lacks advanced features like token expiration or granular permission control, which are essential for FeedApp.

In this experiment, OAuth 2.0 is evaluated against API Keys and Basic Authentication to determine whether its complexity and capabilities justify its selection for FeedApp.

\subsection{Experiment Design}

\subsubsection*{Hypotheses}
This experiment evaluates OAuth 2.0's advantages over API Keys and Basic Authentication in the context of FeedApp’s requirements for secure, scalable, and flexible authorization. The comparison focuses on implementation complexity, security, performance, and scalability.

Hypotheses:


\begin{enumerate}
    \item \textbf{H1}: OAuth 2.0 provides superior security compared to API Keys and Basic Authentication, as it supports token expiration, revocation, and scoped access.

    \item \textbf{H2}: OAuth 2.0 demonstrates better scalability under high concurrent loads due to its token-based design and reduced reliance on static credentials.

    \item \textbf{H3}: OAuth 2.0 is more complex to implement than API Keys or Basic Authentication but provides better long-term maintainability and integration with external systems.
\end{enumerate}

Experimental Setup: Three identical prototypes are implemented, each securing the FeedApp API using a different authentication method:
\medskip

\noindent
Prototype A (OAuth 2.0):
Users authenticate using OAuth 2.0, and the client receives an Access Token with predefined scopes.
Prototype B (API Keys):
Users are assigned a static API Key that must be included in every request.
Prototype C (Basic Authentication):
Users authenticate with their username and password included in every request’s header.
Each prototype is tested under the following scenarios:
\medskip

\noindent
User login and token/key issuance.
Secured API calls, such as creating surveys and casting votes.
Concurrent requests simulating high user loads.
Evaluation Metrics:
\medskip

\noindent

Ease of Implementation:
\begin{itemize}
	\item  Time and effort required to implement each technology.
	\item  Developer feedback on challenges during configuration.
\end{itemize}
Security:
\begin{itemize}
	\item Protection against vulnerabilities (e.g., token theft, replay attacks).
	\item Support for advanced features like token expiration and revocation.
\end{itemize}
Performance:
\begin{itemize}
	\item  Latency in login flows and API requests.
	\item  System throughput under high-load scenarios.
\end{itemize}
Scalability:
\begin{itemize}
	\item  Handling of concurrent requests and performance consistency.
\end{itemize}
\subsection{Experiment Evaluation}

% Write about the results of your experiments, either via personal experience reports, quantitative benchmarks, a demonstrator case study, or a combination of multiple approaches.
The results are analyzed based on the defined metrics, focusing on how OAuth 2.0 compares to the alternatives in the context of FeedApp.

%For some reports, you may have to include a table with experimental
%results or other kinds of tables that, for instance, compare technologies. 
Table~\ref{tab:results} gives an example of how to create a table.

\begin{table}[bth]
	\centering
	\begin{tabular}{llrrrrrr}
		Metric & OAuth 2.0 & API Keys& Basic Authentication
		\\ \hline \hline
		Login Flow Latency & 00ms   &   00ms  &   00ms \\
		API Call Latency & 00ms   &   00ms  &   00ms \\
		Scalability (10k users) & 00\% success   & 00\% success  &  00\% success \\
		Security Score & High   &   Low  &   Very Low \\
		\hline \hline
	\end{tabular}
	\caption{Selected experimental results on the communication protocol example.}
	\label{tab:results}
\end{table}



\section{Prototype Implementation}
\label{sec:implementation}


\section{Conclusions}

The FeedApp project has been a valuable learning experience in understanding and comparing different authentication technologies, specifically OAuth 2.0 and Basic Authentication. The goal was to evaluate these methods in terms of scalability, security, and how easy or difficult they are to implement. By exploring these technologies through practical experiments, the project has shown how important it is to choose the right authentication method for an application, depending on its requirements.

OAuth 2.0 stood out as a powerful and modern authentication framework. It provides advanced security features like token expiration, scope limitations, and stateless validation, which make it very scalable and suitable for large systems. However, implementing OAuth 2.0 requires more effort and a steeper learning curve, especially for someone new to the framework. Thankfully, its documentation is thorough and supported by an active community, which helped during the project. In comparison, Basic Authentication is much simpler and quicker to set up, making it a good choice for small applications or systems that don’t handle a lot of users. However, it doesn’t scale well under heavy loads and lacks the advanced security features that OAuth 2.0 offers.

The process of working on this project also highlighted how the maturity of a technology impacts its usability. OAuth 2.0 has evolved over the years and is now widely used in the industry, making it a reliable and future-proof option for modern applications. In comparison Basic Authentication feels outdated for applications that require strong security and scalability.

The references used throughout the project, including the official OAuth 2.0 specification and documentation from Spring Security, were essential in understanding the technologies and their practical applications. These resources, along with case studies and comparisons from the industry, offer a strong foundation for anyone interested in learning more about authentication.

Overall, this project has shown that while Basic Authentication is simple and straightforward, OAuth 2.0 is the better choice for applications that need to scale or handle sensitive data securely. The practical experience gained through implementing and testing these technologies has provided valuable insights into their strengths and limitations, which will be useful for similar projects in the future.

\bibliographystyle{plain}
\bibliography{report.bib}{}

\end{document}
